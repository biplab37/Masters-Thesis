\documentclass[a4paper,crop=false]{standalone}
% \usepackage{amsmath}
% \usepackage{amssymb}
% \usepackage{fullpage}
% \usepackage[compat=1.1.0]{tikz-feynman}
% \usepackage{wrapfig}
% \usepackage{hyperref}
% \usepackage{graphicx,subcaption,caption}
\begin{document}
    Rather than introducing a constant interaction which couples inter and intra valley fields, we can introduce a scalar field. The interaction is of the form
    \begin{equation}
        -ie(\Psi_{1,k}^{\dagger}\Psi_{1,k+q}\phi_{q} + \Psi_{2,k}^{\dagger}\Psi_{2,k+q}\phi_{q})
    \end{equation}
    To get Coulomb like interaction (i.e. the scalar field behaving like a photon field) we will assume the scalar field propagator to be $\frac{2\pi}{|q|}$. Feynman diagrams for the theory are shown in figure \ref{fig:feynman_scalar}
    \begin{figure}[h]
        \centering
        \documentclass{standalone}

\begin{document}
	\centering
	\begin{subfigure}[t]{0.45\textwidth}
		\begin{tikzpicture}
			\begin{feynman}
				\vertex (a);
				\vertex [right=of a] (b);
				\vertex [above=of a] (c);
				\vertex [right=of c] (d);
				\diagram*{
					(a)--[boson](b);
					(c)--[fermion](d);
				};
			\end{feynman}
		\end{tikzpicture}
		\caption{electron and photon propagator}
	\end{subfigure}
	\begin{subfigure}[t]{0.3\textwidth}
		\begin{tikzpicture}
			\begin{feynman}
				\vertex (a);
				\vertex [right=of a] (p);
				\vertex [above left=of a] (f);
				\vertex [below left=of a] (i);
				\diagram*{
					(i)--[fermion](a)--[fermion](f);
					(a)--[boson](p);
				};
	 		\end{feynman}
		\end{tikzpicture}
		\caption{vertex}
	\end{subfigure}

\end{document}
        \caption{feynman diagrams for scalar coupling}
        \label{fig:feynman_scalar}
    \end{figure}
    \subsection{Two point function}
        \begin{figure}[h]
            \centering
            \begin{subfigure}[t]{0.49\textwidth}
                \documentclass{standalone}
% \usepackage[compat=1.1.0]{tikz-feynman}

\begin{document}
	\begin{tikzpicture}
        \begin{feynman}
            \vertex (m);
            \vertex [left=of m] (a);
            \vertex [right=of m] (b);
            \vertex [above=of m] (c);
            \vertex [above=of c] (d)[label=below:\(k'\)];
            \diagram*{
                (a)--[fermion,edge label=\(k\)](m)--[fermion,edge label=\(k\)](b);
                (m)--[boson,edge label=\(q\)](c);
                (c)--[fermion,out=120,in=180](d)--[fermion,out=0,in=60](c);
            };
        \end{feynman}
    \end{tikzpicture}
\end{document}
                \caption{Tadpole Diagram}
                \label{tadpole}
            \end{subfigure}
            \begin{subfigure}[t]{0.49\textwidth}
                \documentclass{standalone}
% \usepackage[compat=1.1.0]{tikz-feynman}

\begin{document}
    \begin{tikzpicture}
        \begin{feynman}
            \vertex (a);
            \vertex [left=of a] (i);
            \vertex [right=of a] (b);
            \vertex [right=of b] (f);
            \diagram*{
                (i)--[fermion,edge label=\(k\)](a)--[fermion,edge label'=\(k-q\)](b)--[fermion,edge label=\(k\)](f);
                (a)--[boson,edge label=\(q\),out=90,in=90](b);
            };
        \end{feynman}
    \end{tikzpicture}
\end{document}
                \caption{Electron Self Energy}
                \label{electron_self_energy}
            \end{subfigure}
        \end{figure}
        \subsubsection*{Tadpole Diagram}
            In the tadpole diagram \ref{tadpole} momentum consevation at the vertices gives $q=0$. This term known as Hartree term usually gets cancelled by the background uniform charge distribution.
        \subsubsection*{Electron Self Energy}
            This diagram \ref{electron_self_energy} translates to 
            \begin{equation}
                (-ie)^2\int^{\infty}_{-\infty}\frac{d\omega}{2\pi}\int_{\frac{\Lambda}{s}<|q|<\Lambda}\frac{d^2q}{(2\pi)^2} G^{(0)}_1(i\omega,k - q)\frac{2\pi}{|q|}
            \end{equation}
            the $\omega$ integral is straightforward and gives following
            \begin{equation}
                -e^2\int_{\frac{\Lambda}{s}<|q|<\Lambda}\frac{d^2q}{4\pi} (1 + \frac{\sigma . (k-q)}{|k-q|})\frac{1}{|q|}
            \end{equation}
            the q integral is difficult to do and analytic result can be obtained in Mathematica. Although as we are not interested in the higher powers of external momentum k we can expand the integrand in powers of k and then get the integral easily. For example at order k we get $\frac{\sigma . k e^2}{2} log(s)$ which renormalises $v_F$.
    \subsection{Photon Self Energy}
        \begin{wrapfigure}{r}{0.3\textwidth}
            \centering
            \documentclass{standalone}

\begin{document}
	\begin{tikzpicture}
		\begin{feynman}
			\vertex (b);
			\vertex [left=of b] (a);
			\vertex [right=of b] (c);
			\vertex [right=of c] (d);
			\diagram*{
				(a)--[boson,edge label=\(\omega k\)](b);
				(b)--[fermion,out=90,in=90,edge label=\(\omega_1 k_1\)](c);
				(c)--[fermion,out=-90,in=-90,edge label=\(\omega_1-\omega k_1-k\)](b);
				(c)--[boson,edge label=\(\omega k\)](d);
			};

		\end{feynman}
	\end{tikzpicture}
\end{document}
            \caption{Photon Self Energy}
            \label{fig:photon_self_energy}
        \end{wrapfigure}
        With the vertex \ref{fig:feynman_scalar} one can form the diagram \ref{fig:photon_self_energy} which will give correction to the photon propagator. The correction is given by
        \begin{equation}
            e^2\int_{\frac{\Lambda}{s}<|k|<\Lambda}\frac{d^2k}{(2\pi)^2}\int_{-\infty}^{\infty}\frac{d\omega}{2\pi}G_{1}^{(0)}(i\omega_1,k_1)G_{1}^{(0)}(i(\omega_1-\omega),k_1-k)
        \end{equation}
        The $\omega$ integral can be evaluated as given in the appendix (see \ref{eqn:I2prime}).

        \subsection{Vertex Correction}
        \begin{wrapfigure}{l}{0.3\textwidth}
            \centering
            \documentclass{standalone}
% \usepackage[compat=1.1.0]{tikz-feynman}

\begin{document}
	\begin{tikzpicture}
        \begin{feynman}
            \vertex (a);
            \vertex [right=of a] (b){q};
            \vertex [above left=of a] (c);
            \vertex [below left=of a] (d);
            \vertex [above left=of c] (e);
            \vertex [below left=of d] (f);
            \diagram*{
                (f)--[fermion,edge label=\(k+q\)](d)--[fermion,edge label'=\(k+q-q_1\)](a)--[fermion,edge label'=\(k-q_1\)](c)--[fermion,edge label=\(k\)](e);
                (c)--[boson,edge label'=\(q_1\)](d);
                (a)--[boson](b);
            };
        \end{feynman}
    \end{tikzpicture}
\end{document}
            \caption{Vertex Correction}
            \label{fig:vertex_corection}
        \end{wrapfigure}   
        Similarly one can calculate the vertex correction \ref{fig:vertex_corection} as well
        \begin{equation}
            e^2\int_{\frac{\Lambda}{s}<|q_1|<\Lambda}\frac{d^2q_1}{(2\pi)^2}\int_{-\infty}^{\infty}\frac{d\omega}{2\pi}G_{1}^{(0)}(i\omega_1,k + q + q_1)G_{1}^{(0)}(i(\omega_1-\omega),k-q_1)\frac{2\pi}{|q_1|}
        \end{equation}  

\end{document}