\documentclass[a4paper]{article}

%%% Packages %%%
\usepackage{amsmath}
\usepackage{amssymb}
\usepackage{fullpage}
\usepackage[compat=1.1.0]{tikz-feynman}
\usepackage{wrapfig}
\usepackage{hyperref}
\usepackage{graphicx,subcaption,caption}
\usepackage[most]{tcolorbox}
\usepackage{subfiles}
\usepackage{standalone}
\newtcolorbox{mybox}[2][]{%
  attach boxed title to top center
               = {yshift=-8pt},
  colback      = white,
  colframe     = black,
  fonttitle    = \bfseries,
  colbacktitle = black,
  title        = #2,#1,
  enhanced,
}

\title{
	Oneloop
}
\author{Biplab Mahato}
\begin{document}
    
    \tableofcontents
    
    \section{Constant Interaction}
    \subsection{Free Hamiltonian}
        \subfile{subfile/Constant_Interaction/free_hamiltonian.tex}
    \subsection{Renormalisation of two point function}
        \subfile{subfile/Constant_Interaction/two_point_renormalisation.tex}
    \subsection{Renormalisation of interaction}
        \subfile{subfile/Constant_Interaction/interaction_renormalisation.tex}
    \subsection{Turning intervalley potential on}
        \subfile{subfile/Constant_Interaction/intervalley_potential.tex}
    \subsection{Two point function second order}
        \subfile{subfile/Constant_Interaction/twopoint.tex}
    \subsection{Conclusion}
        \begin{itemize}
            \item $v_F$ does not renormalise as we change the cutoff momentum. To renormalise $v_F$ we need the interaction to have momentum dependence. This is easy to see. Consider the interaction of the form $\frac{V}{2!2!} \Psi^{\dagger}_{1,k_4}\Psi^{\dagger}_{1,k_3}\Psi_{1,k_2}\Psi_{1,k_1}$. TO renormalise $v_F$ we have to contract two momentum to get free particle green functions. This introduces delta function on the momentum which gives $\frac{V}{2!2!} \Psi^{\dagger}_{1,k_1}G^{(0)}_1(i\omega,k)\Psi_{1,k_1}$ with integration over $k$ and $\omega$. This will always give $\Psi^{\dagger}_{1,k_1}(constant)\Psi_{1,k_1}$ unless $V$ is a function of both $k$ and $k_1$.
            \item a chemical potential like term (mass like) arises due to renormalisation.
            \begin{equation}
                \mu' = s\left( \mu + \frac{(V_1+V_2)\Lambda^2}{8\pi}\left(1 - \frac{1}{s^2} \right) \right)
            \end{equation}
            \item $V_1$ does not renormalise under $\Lambda \to \frac{\Lambda}{s}$ transformation.
            \item $V_2$ renormalises to
            \begin{equation}
                V'_2 = V_2 - ZS' - \frac{1}{2}BCS
                     = V_2 -\frac{1}{2}\frac{V^2_2\Lambda}{8\pi v_F}\left(1- \frac{1}{s}\right)
            \end{equation}
            As $ZS' = - BCS$.

        \end{itemize}
    \section{Scalar Coupling}
        \subfile{subfile/Scalar_Coupling/scalar_coupling.tex}
\end{document}